\newcommand{\NWtarget}[2]{#2}
\newcommand{\NWlink}[2]{#2}
\newcommand{\NWtxtMacroDefBy}{Fragment defined by}
\newcommand{\NWtxtMacroRefIn}{Fragment referenced in}
\newcommand{\NWtxtMacroNoRef}{Fragment never referenced}
\newcommand{\NWtxtDefBy}{Defined by}
\newcommand{\NWtxtRefIn}{Referenced in}
\newcommand{\NWtxtNoRef}{Not referenced}
\newcommand{\NWtxtFileDefBy}{File defined by}
\newcommand{\NWtxtIdentsUsed}{Uses:}
\newcommand{\NWtxtIdentsNotUsed}{Never used}
\newcommand{\NWtxtIdentsDefed}{Defines:}
\newcommand{\NWsep}{${\diamond}$}
\newcommand{\NWnotglobal}{(not defined globally)}
\newcommand{\NWuseHyperlinks}{}
% 2015-12-25
% It's been a while since I last wrote something in Latex.
%
% This is going to be an exploration of literate programming with nuweb.

\documentclass[a4paper]{report}
\newif\ifshowcode
\showcodetrue

\usepackage{latexsym}
\usepackage{listings}
\usepackage{./nwhren}

\usepackage{color}
\definecolor{linkcolor}{rgb}{0, 0, 0.7}
\definecolor{light}{gray}{.93}

\usepackage[%
backref,%
raiselinks,%
pdfhighlight=/O,%
pagebackref,%
hyperfigures,%
breaklinks,%
colorlinks,%
pdfpagemode=UseNone,%
pdfstartview=FitBH,%
linkcolor={linkcolor},%
anchorcolor={linkcolor},%
citecolor={linkcolor},%
filecolor={linkcolor},%
menucolor={linkcolor},%
urlcolor={linkcolor}%
]{hyperref}

\setlength{\oddsidemargin}{0cm}
\setlength{\evensidemargin}{0cm}
\setlength{\topmargin}{0cm}
\addtolength{\topmargin}{-\headheight}
\addtolength{\topmargin}{-\headsep}
\setlength{\textheight}{22.6cm}
\setlength{\textwidth}{16.51cm}
\setlength{\marginparwidth}{1.27cm}

\title{An Exploration of Literate Programming\\ With Nuweb and Go}
\date{}
\author{smotti}

\begin{document}

\pagenumbering{roman}
\begin{titlepage}
\maketitle
\end{titlepage}

\tableofcontents

\chapter{Hello World}
\pagenumbering{arabic}

In Go we first have to define the name of a package to which the file and its
content belongs to. The package \textit{main}, that we use, will instruct the
Go compiler to build an executable. Other package names indicate that it's a
shared library.
\begin{flushleft} \small
\begin{minipage}{\linewidth}\label{scrap1}\raggedright\small
\NWtarget{nuweb1a}{} \verb@"hello_world.go"@\nobreak\ {\footnotesize {1a}}$\equiv$
\vspace{-1ex}
\begin{list}{}{} \item
\mbox{}\lstinline@package main@\\
\mbox{}\lstinline@@\hbox{$\langle\,${\itshape Import packages}\nobreak\ {\footnotesize \NWlink{nuweb1b}{1b}}$\,\rangle$}\lstinline@@\\
\mbox{}\lstinline@@\hbox{$\langle\,${\itshape Hello World func}\nobreak\ {\footnotesize \NWlink{nuweb2}{2}}$\,\rangle$}\lstinline@@\\
\mbox{}\lstinline@@\hbox{$\langle\,${\itshape Main func}\nobreak\ {\footnotesize \NWlink{nuweb1c}{1c}}$\,\rangle$}\lstinline@@\\
\mbox{}\lstinline@@{\NWsep}
\end{list}
\vspace{-1.5ex}
\footnotesize
\begin{list}{}{\setlength{\itemsep}{-\parsep}\setlength{\itemindent}{-\leftmargin}}
\item \NWtxtIdentsUsed\nobreak\  \verb@main@\nobreak\ \NWlink{nuweb1c}{1c}.
\item{}
\end{list}
\end{minipage}\vspace{4ex}
\end{flushleft}
After that we import packages from the standard library that provide the
required functionality (i.e. printing to stdout etc.).
\begin{flushleft} \small
\begin{minipage}{\linewidth}\label{scrap2}\raggedright\small
\NWtarget{nuweb1b}{} $\langle\,${\itshape Import packages}\nobreak\ {\footnotesize {1b}}$\,\rangle\equiv$
\vspace{-1ex}
\begin{list}{}{} \item
\mbox{}\lstinline@import (@\\
\mbox{}\lstinline@  "fmt"@\\
\mbox{}\lstinline@)@\\
\mbox{}\lstinline@@{\NWsep}
\end{list}
\vspace{-1.5ex}
\footnotesize
\begin{list}{}{\setlength{\itemsep}{-\parsep}\setlength{\itemindent}{-\leftmargin}}
\item \NWtxtMacroRefIn\ \NWlink{nuweb1a}{1a}.

\item{}
\end{list}
\end{minipage}\vspace{4ex}
\end{flushleft}
Now me implement the \emph{main} function of our small program. A function in
Go is declared with \emph{func} followed by the function name, an argument
list and a list of value types the function returns. The main function in Go
takes no arguments and returns no value.
\begin{flushleft} \small
\begin{minipage}{\linewidth}\label{scrap3}\raggedright\small
\NWtarget{nuweb1c}{} $\langle\,${\itshape Main func}\nobreak\ {\footnotesize {1c}}$\,\rangle\equiv$
\vspace{-1ex}
\begin{list}{}{} \item
\mbox{}\lstinline@func main() {@\\
\mbox{}\lstinline@  helloWorld()@\\
\mbox{}\lstinline@}@\\
\mbox{}\lstinline@@{\NWsep}
\end{list}
\vspace{-1.5ex}
\footnotesize
\begin{list}{}{\setlength{\itemsep}{-\parsep}\setlength{\itemindent}{-\leftmargin}}
\item \NWtxtMacroRefIn\ \NWlink{nuweb1a}{1a}.
\item \NWtxtIdentsDefed\nobreak\  \verb@main@\nobreak\ \NWlink{nuweb1a}{1a}.\item \NWtxtIdentsUsed\nobreak\  \verb@helloWorld@\nobreak\ \NWlink{nuweb2}{2}.
\item{}
\end{list}
\end{minipage}\vspace{4ex}
\end{flushleft}
\newpage
Just to have something to write we implement a function to just print \emph{
Hello World} to \textit{stdout}. 
\begin{flushleft} \small
\begin{minipage}{\linewidth}\label{scrap4}\raggedright\small
\NWtarget{nuweb2}{} $\langle\,${\itshape Hello World func}\nobreak\ {\footnotesize {2}}$\,\rangle\equiv$
\vspace{-1ex}
\begin{list}{}{} \item
\mbox{}\lstinline@func helloWorld() {@\\
\mbox{}\lstinline@  fmt.Println("Hello World")@\\
\mbox{}\lstinline@}@\\
\mbox{}\lstinline@@{\NWsep}
\end{list}
\vspace{-1.5ex}
\footnotesize
\begin{list}{}{\setlength{\itemsep}{-\parsep}\setlength{\itemindent}{-\leftmargin}}
\item \NWtxtMacroRefIn\ \NWlink{nuweb1a}{1a}.
\item \NWtxtIdentsDefed\nobreak\  \verb@helloWorld@\nobreak\ \NWlink{nuweb1c}{1c}.
\item{}
\end{list}
\end{minipage}\vspace{4ex}
\end{flushleft}
\chapter{Indexes}


{\small\begin{list}{}{\setlength{\itemsep}{-\parsep}\setlength{\itemindent}{-\leftmargin}}
\item \verb@"hello_world.go"@ {\footnotesize {\NWtxtDefBy} \NWlink{nuweb1a}{1a}.}
\end{list}}

{\small\begin{list}{}{\setlength{\itemsep}{-\parsep}\setlength{\itemindent}{-\leftmargin}}
\item $\langle\,$Hello World func\nobreak\ {\footnotesize \NWlink{nuweb2}{2}}$\,\rangle$ {\footnotesize {\NWtxtRefIn} \NWlink{nuweb1a}{1a}.}
\item $\langle\,$Import packages\nobreak\ {\footnotesize \NWlink{nuweb1b}{1b}}$\,\rangle$ {\footnotesize {\NWtxtRefIn} \NWlink{nuweb1a}{1a}.}
\item $\langle\,$Main func\nobreak\ {\footnotesize \NWlink{nuweb1c}{1c}}$\,\rangle$ {\footnotesize {\NWtxtRefIn} \NWlink{nuweb1a}{1a}.}
\end{list}}

{\small\begin{list}{}{\setlength{\itemsep}{-\parsep}\setlength{\itemindent}{-\leftmargin}}
\item \verb@helloWorld@: \NWlink{nuweb1c}{1c}, \underline{\NWlink{nuweb2}{2}}.
\item \verb@main@: \NWlink{nuweb1a}{1a}, \underline{\NWlink{nuweb1c}{1c}}.
\end{list}}

\end{document}
